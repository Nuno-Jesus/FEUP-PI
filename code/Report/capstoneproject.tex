\documentclass[10pt]{article}
\usepackage[portuguese]{babel} % para indice em português
\usepackage{url}
\usepackage{graphicx}
\graphicspath{ {./figures/} }



\begin{document}

\begin{titlepage}
	\begin{center}
		\vspace*{1cm}
		
    {\fontsize{17}{16}\selectfont \textbf{Desenvolvimento de uma aplicação de aprendizagem digital gamificada}}
     
		
		\vspace{0.5cm}
		Relatório Final do Projeto Integrador
		
		\vspace{1.5cm}
		
		\textbf{André Moreira Leal Leonor}\\
		\textbf{João Pedro Brito Veloso}\\
		\textbf{Luís Guilherme de Melo Félix Diogo}\\
		\textbf{Nuno Miguel Carvalho de Jesus}
		
				\vfill
		
		\includegraphics[width=0.4\textwidth]{UPORTO_fundotransparente}
		
		\vfill
		
	
		
		Licenciatura em Engenharia Informática e Computação
		
		\vspace{0.8cm}
		
		
		\textbf{Tutor na U.Porto}: António Coelho\\
		\textbf{Orientador na empresa/Proponente}: Bárbara Andrez \\ %Se diferente do tutor

\vspace{0.4cm}
		15 de Junho de 2023
		
	\end{center}
\end{titlepage}


\thispagestyle{empty} 
\clearpage

\thispagestyle{empty}
\tableofcontents

 \clearpage
 
\section{Introdução}
\subsection{Enquadramento}
Este relatório tem como objetivo apresentar o desenvolvimento de um projeto que visa integrar uma pesquisa no âmbito de um doutoramento em Informação e Comunicação em Plataformas Digitais. O projeto foi realizado em colaboração entre a Universidade do Porto (FLUP) e a Universidade de Aveiro, como parte de um curso conjunto. 

O contexto organizacional em que este projeto foi conduzido é o Museu Militar do Porto, uma instituição cultural que busca promover a compreensão da história militar e contribuir para a aprendizagem dos visitantes. Reconhecendo as possibilidades oferecidas pelas soluções digitais e cientes dos desafios envolvidos na implementação de aplicativos móveis em museus, o Museu Militar do Porto decidiu explorar a introdução de mecânicas de jogo para melhorar a experiência dos visitantes e estabelecer conexões mais significativas entre os processos de informação e a aprendizagem digital.

O problema abordado neste projeto é a necessidade de compreender como os processos de informação e comunicação podem ser aprimorados por meio de estratégias de gamificação em uma aplicação móvel construída especificamente para o contexto do museu. Com investimentos significativos envolvidos na implementação de soluções digitais, é fundamental entender como os elementos de jogos, como pontos, desafios, mini-jogos, narrativa envolvente, personagens animadas e realidade aumentada, podem ser aplicados para melhorar a experiência de aprendizagem dos visitantes.

A motivação para este trabalho é a busca por soluções inovadoras que facilitem a aprendizagem em museus, utilizando a tecnologia móvel e a realidade aumentada como ferramentas. Além disso, espera-se que esse projeto sirva como exemplo para o desenvolvimento futuro de ações digitais em museus, adaptáveis a diferentes grupos e necessidades. A proposta também oferece uma oportunidade para o Museu Militar do Porto aumentar sua visibilidade, tanto pela temática incomum abordada quanto pelo trabalho realizado durante o desenvolvimento do protótipo.

\subsection{Objetivos e resultados esperados}

Indicar os objetivos do trabalho e resultados esperados.

\subsection{Estrutura do relatório}

Explicar a estrutura do relatório.




\section{Metodologia utilizada e principais atividades desenvolvidas}

\subsection{Metodologia utilizada}

    Descrever a metodologia\cite{despa2014comparative} seguida (exemplo: desenvolvimento iterativo com sprints quinzenais e reuniões semanais de acompanhamento) e recursos utilizados (exemplo: GitHub \cite{github}, etc.).
	
	
\subsection{Intervenientes, papéis e responsabilidades}

Identificar a equipa de projeto, stakeholders e outros invernientes com os quais existiu interação; no caso de trabalho em grupo, clarificar os papéis e responsabilidades de cada elemento do grupo.

\subsection{Atividades desenvolvidas}

Descrever as atividades realizadas ao londo do tempo (incluindo eventos relevantes, como apresentações, reuniões com clientes, etc.) e respectivos deliverables, recorrendo tipicamente a um diagrama de Gantt\cite{gantt} e a uma descrição sumária de cada atividade/deliverable. 


\section{Desenvolvimento da solução}

\subsection{Requisitos}

Identificar os requisitos funcionais e não funcionais relevantes e respectivas fontes, bem como restrições ao projeto.

\subsection{Arquitetura e tecnologias}

Arquitetura e tecnologias utilizadas e respetiva justificação, diagramas técnicos elabordos, dificuldades técnicas encontradas e sua resolução, etc.

\subsection{Solução desenvolvida}

Apresentar a solução desenvolvida na ótica do utilizador, com ajuda de screenshots.

\subsection{Validação}

Descrição da validação da solução desenvolvida (por exemplo, resultados de avaliação experimental, testes efetuados, feedback de utilizadores ou especialistas, etc.).


\section{Conclusões}


	
\subsection{Resultados alcançados} 	
 Sumariar os resultados alcançados e contribuições (em relação aos objetivos).
	
 No caso de trabalho em grupo, clarificar as contribuições individuais, em termos qualitativos e quantitativos (percentagem).
\subsection{Lições aprendidas} 	
 Refletir sobre as lições aprendidas (tendo em conta os objetivos de aprendizagem).
	
\subsection{Trabalho futuro} 
 
 Ideias de melhorias e trabalho futuro. 
	


\bibliographystyle{plain} 
\bibliography{refs} % Entries are in the refs.bib file
 
\end{document}